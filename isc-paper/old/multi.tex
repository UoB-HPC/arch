
%% bare_conf.tex
%% V1.4b
%% 2015/08/26
%% by Michael Shell
%% See:
%% http://www.michaelshell.org/
%% for current contact information.
%%
%% This is a skeleton file demonstrating the use of IEEEtran.cls
%% (requires IEEEtran.cls version 1.8b or later) with an IEEE
%% conference paper.
%%
%% Support sites:
%% http://www.michaelshell.org/tex/ieeetran/
%% http://www.ctan.org/pkg/ieeetran
%% and
%% http://www.ieee.org/

%%*************************************************************************
%% Legal Notice:
%% This code is offered as-is without any warranty either expressed or
%% implied; without even the implied warranty of MERCHANTABILITY or
%% FITNESS FOR A PARTICULAR PURPOSE! 
%% User assumes all risk.
%% In no event shall the IEEE or any contributor to this code be liable for
%% any damages or losses, including, but not limited to, incidental,
%% consequential, or any other damages, resulting from the use or misuse
%% of any information contained here.
%%
%% All comments are the opinions of their respective authors and are not
%% necessarily endorsed by the IEEE.
%%
%% This work is distributed under the LaTeX Project Public License (LPPL)
%% ( http://www.latex-project.org/ ) version 1.3, and may be freely used,
%% distributed and modified. A copy of the LPPL, version 1.3, is included
%% in the base LaTeX documentation of all distributions of LaTeX released
%% 2003/12/01 or later.
%% Retain all contribution notices and credits.
%% ** Modified files should be clearly indicated as such, including  **
%% ** renaming them and changing author support contact information. **
%%*************************************************************************


% *** Authors should verify (and, if needed, correct) their LaTeX system  ***
% *** with the testflow diagnostic prior to trusting their LaTeX platform ***
% *** with production work. The IEEE's font choices and paper sizes can   ***
% *** trigger bugs that do not appear when using other class files.       ***                          ***
% The testflow support page is at:
% http://www.michaelshell.org/tex/testflow/



\documentclass[conference]{IEEEtran}
% Some Computer Society conferences also require the compsoc mode option,
% but others use the standard conference format.
%
% If IEEEtran.cls has not been installed into the LaTeX system files,
% manually specify the path to it like:
% \documentclass[conference]{../sty/IEEEtran}


\usepackage{listings}

\lstdefinelanguage{ptx}
{morekeywords={mul, ld, global, nc, rn, f64, add, fma, sub, rn, s64, setp, lt, bra, st},}

\usepackage{courier}

\lstset{basicstyle=\small\ttfamily,breaklines=true}

% Some very useful LaTeX packages include:
% (uncomment the ones you want to load)


% *** MISC UTILITY PACKAGES ***
%
%\usepackage{ifpdf}
% Heiko Oberdiek's ifpdf.sty is very useful if you need conditional
% compilation based on whether the output is pdf or dvi.
% usage:
% \ifpdf
%   % pdf code
% \else
%   % dvi code
% \fi
% The latest version of ifpdf.sty can be obtained from:
% http://www.ctan.org/pkg/ifpdf
% Also, note that IEEEtran.cls V1.7 and later provides a builtin
% \ifCLASSINFOpdf conditional that works the same way.
% When switching from latex to pdflatex and vice-versa, the compiler may
% have to be run twice to clear warning/error messages.






% *** CITATION PACKAGES ***
%
\usepackage{cite}
% cite.sty was written by Donald Arseneau
% V1.6 and later of IEEEtran pre-defines the format of the cite.sty package
% \cite{} output to follow that of the IEEE. Loading the cite package will
% result in citation numbers being automatically sorted and properly
% "compressed/ranged". e.g., [1], [9], [2], [7], [5], [6] without using
% cite.sty will become [1], [2], [5]--[7], [9] using cite.sty. cite.sty's
% \cite will automatically add leading space, if needed. Use cite.sty's
% noadjust option (cite.sty V3.8 and later) if you want to turn this off
% such as if a citation ever needs to be enclosed in parenthesis.
% cite.sty is already installed on most LaTeX systems. Be sure and use
% version 5.0 (2009-03-20) and later if using hyperref.sty.
% The latest version can be obtained at:
% http://www.ctan.org/pkg/cite
% The documentation is contained in the cite.sty file itself.






% *** GRAPHICS RELATED PACKAGES ***
%
\ifCLASSINFOpdf
  % \usepackage[pdftex]{graphicx}
  % declare the path(s) where your graphic files are
  % \graphicspath{{../pdf/}{../jpeg/}}
  % and their extensions so you won't have to specify these with
  % every instance of \includegraphics
  % \DeclareGraphicsExtensions{.pdf,.jpeg,.png}
\else
  % or other class option (dvipsone, dvipdf, if not using dvips). graphicx
  % will default to the driver specified in the system graphics.cfg if no
  % driver is specified.
  % \usepackage[dvips]{graphicx}
  % declare the path(s) where your graphic files are
  % \graphicspath{{../eps/}}
  % and their extensions so you won't have to specify these with
  % every instance of \includegraphics
  % \DeclareGraphicsExtensions{.eps}
\fi
% graphicx was written by David Carlisle and Sebastian Rahtz. It is
% required if you want graphics, photos, etc. graphicx.sty is already
% installed on most LaTeX systems. The latest version and documentation
% can be obtained at: 
% http://www.ctan.org/pkg/graphicx
% Another good source of documentation is "Using Imported Graphics in
% LaTeX2e" by Keith Reckdahl which can be found at:
% http://www.ctan.org/pkg/epslatex
%
% latex, and pdflatex in dvi mode, support graphics in encapsulated
% postscript (.eps) format. pdflatex in pdf mode supports graphics
% in .pdf, .jpeg, .png and .mps (metapost) formats. Users should ensure
% that all non-photo figures use a vector format (.eps, .pdf, .mps) and
% not a bitmapped formats (.jpeg, .png). The IEEE frowns on bitmapped formats
% which can result in "jaggedy"/blurry rendering of lines and letters as
% well as large increases in file sizes.
%
% You can find documentation about the pdfTeX application at:
% http://www.tug.org/applications/pdftex





% *** MATH PACKAGES ***
%
%\usepackage{amsmath}
% A popular package from the American Mathematical Society that provides
% many useful and powerful commands for dealing with mathematics.
%
% Note that the amsmath package sets \interdisplaylinepenalty to 10000
% thus preventing page breaks from occurring within multiline equations. Use:
%\interdisplaylinepenalty=2500
% after loading amsmath to restore such page breaks as IEEEtran.cls normally
% does. amsmath.sty is already installed on most LaTeX systems. The latest
% version and documentation can be obtained at:
% http://www.ctan.org/pkg/amsmath





% *** SPECIALIZED LIST PACKAGES ***
%
%\usepackage{algorithmic}
% algorithmic.sty was written by Peter Williams and Rogerio Brito.
% This package provides an algorithmic environment fo describing algorithms.
% You can use the algorithmic environment in-text or within a figure
% environment to provide for a floating algorithm. Do NOT use the algorithm
% floating environment provided by algorithm.sty (by the same authors) or
% algorithm2e.sty (by Christophe Fiorio) as the IEEE does not use dedicated
% algorithm float types and packages that provide these will not provide
% correct IEEE style captions. The latest version and documentation of
% algorithmic.sty can be obtained at:
% http://www.ctan.org/pkg/algorithms
% Also of interest may be the (relatively newer and more customizable)
% algorithmicx.sty package by Szasz Janos:
% http://www.ctan.org/pkg/algorithmicx



% *** ALIGNMENT PACKAGES ***
%
%\usepackage{array}
% Frank Mittelbach's and David Carlisle's array.sty patches and improves
% the standard LaTeX2e array and tabular environments to provide better
% appearance and additional user controls. As the default LaTeX2e table
% generation code is lacking to the point of almost being broken with
% respect to the quality of the end results, all users are strongly
% advised to use an enhanced (at the very least that provided by array.sty)
% set of table tools. array.sty is already installed on most systems. The
% latest version and documentation can be obtained at:
% http://www.ctan.org/pkg/array


% IEEEtran contains the IEEEeqnarray family of commands that can be used to
% generate multiline equations as well as matrices, tables, etc., of high
% quality.




% *** SUBFIGURE PACKAGES ***
%\ifCLASSOPTIONcompsoc
%  \usepackage[caption=false,font=normalsize,labelfont=sf,textfont=sf]{subfig}
%\else
%  \usepackage[caption=false,font=footnotesize]{subfig}
%\fi
% subfig.sty, written by Steven Douglas Cochran, is the modern replacement
% for subfigure.sty, the latter of which is no longer maintained and is
% incompatible with some LaTeX packages including fixltx2e. However,
% subfig.sty requires and automatically loads Axel Sommerfeldt's caption.sty
% which will override IEEEtran.cls' handling of captions and this will result
% in non-IEEE style figure/table captions. To prevent this problem, be sure
% and invoke subfig.sty's "caption=false" package option (available since
% subfig.sty version 1.3, 2005/06/28) as this is will preserve IEEEtran.cls
% handling of captions.
% Note that the Computer Society format requires a larger sans serif font
% than the serif footnote size font used in traditional IEEE formatting
% and thus the need to invoke different subfig.sty package options depending
% on whether compsoc mode has been enabled.
%
% The latest version and documentation of subfig.sty can be obtained at:
% http://www.ctan.org/pkg/subfig




% *** FLOAT PACKAGES ***
%
%\usepackage{fixltx2e}
% fixltx2e, the successor to the earlier fix2col.sty, was written by
% Frank Mittelbach and David Carlisle. This package corrects a few problems
% in the LaTeX2e kernel, the most notable of which is that in current
% LaTeX2e releases, the ordering of single and double column floats is not
% guaranteed to be preserved. Thus, an unpatched LaTeX2e can allow a
% single column figure to be placed prior to an earlier double column
% figure.
% Be aware that LaTeX2e kernels dated 2015 and later have fixltx2e.sty's
% corrections already built into the system in which case a warning will
% be issued if an attempt is made to load fixltx2e.sty as it is no longer
% needed.
% The latest version and documentation can be found at:
% http://www.ctan.org/pkg/fixltx2e


%\usepackage{stfloats}
% stfloats.sty was written by Sigitas Tolusis. This package gives LaTeX2e
% the ability to do double column floats at the bottom of the page as well
% as the top. (e.g., "\begin{figure*}[!b]" is not normally possible in
% LaTeX2e). It also provides a command:
%\fnbelowfloat
% to enable the placement of footnotes below bottom floats (the standard
% LaTeX2e kernel puts them above bottom floats). This is an invasive package
% which rewrites many portions of the LaTeX2e float routines. It may not work
% with other packages that modify the LaTeX2e float routines. The latest
% version and documentation can be obtained at:
% http://www.ctan.org/pkg/stfloats
% Do not use the stfloats baselinefloat ability as the IEEE does not allow
% \baselineskip to stretch. Authors submitting work to the IEEE should note
% that the IEEE rarely uses double column equations and that authors should try
% to avoid such use. Do not be tempted to use the cuted.sty or midfloat.sty
% packages (also by Sigitas Tolusis) as the IEEE does not format its papers in
% such ways.
% Do not attempt to use stfloats with fixltx2e as they are incompatible.
% Instead, use Morten Hogholm'a dblfloatfix which combines the features
% of both fixltx2e and stfloats:
%
% \usepackage{dblfloatfix}
% The latest version can be found at:
% http://www.ctan.org/pkg/dblfloatfix




% *** PDF, URL AND HYPERLINK PACKAGES ***
%
%\usepackage{url}
% url.sty was written by Donald Arseneau. It provides better support for
% handling and breaking URLs. url.sty is already installed on most LaTeX
% systems. The latest version and documentation can be obtained at:
% http://www.ctan.org/pkg/url
% Basically, \url{my_url_here}.




% *** Do not adjust lengths that control margins, column widths, etc. ***
% *** Do not use packages that alter fonts (such as pslatex).         ***
% There should be no need to do such things with IEEEtran.cls V1.6 and later.
% (Unless specifically asked to do so by the journal or conference you plan
% to submit to, of course. )

\usepackage{graphicx}


% correct bad hyphenation here
\hyphenation{op-tical net-works semi-conduc-tor}


\begin{document}
%
% paper title
% Titles are generally capitalized except for words such as a, an, and, as,
% at, but, by, for, in, nor, of, on, or, the, to and up, which are usually
% not capitalized unless they are the first or last word of the title.
% Linebreaks \\ can be used within to get better formatting as desired.
% Do not put math or special symbols in the title.
\title{A Time-Dependent Multi-Physics Proxy for Performance Evaluation of Nuclear Reactor Cores}

% author names and affiliations
% use a multiple column layout for up to three different
% affiliations
\author{\IEEEauthorblockN{Matt Martineau, Simon McIntosh-Smith}
\IEEEauthorblockA{University of Bristol, UK\\
Email: \{m.martineau, cssnmis\}@bristol.ac.uk}}

% conference papers do not typically use \thanks and this command
% is locked out in conference mode. If really needed, such as for
% the acknowledgment of grants, issue a \IEEEoverridecommandlockouts
% after \documentclass

% for over three affiliations, or if they all won't fit within the width
% of the page, use this alternative format:
% 
%\author{\IEEEauthorblockN{Michael Shell\IEEEauthorrefmark{1},
%Homer Simpson\IEEEauthorrefmark{2},
%James Kirk\IEEEauthorrefmark{3}, 
%Montgomery Scott\IEEEauthorrefmark{3} and
%Eldon Tyrell\IEEEauthorrefmark{4}}
%\IEEEauthorblockA{\IEEEauthorrefmark{1}School of Electrical and Computer Engineering\\
%Georgia Institute of Technology,
%Atlanta, Georgia 30332--0250\\ Email: see http://www.michaelshell.org/contact.html}
%\IEEEauthorblockA{\IEEEauthorrefmark{2}Twentieth Century Fox, Springfield, USA\\
%Email: homer@thesimpsons.com}
%\IEEEauthorblockA{\IEEEauthorrefmark{3}Starfleet Academy, San Francisco, California 96678-2391\\
%Telephone: (800) 555--1212, Fax: (888) 555--1212}
%\IEEEauthorblockA{\IEEEauthorrefmark{4}Tyrell Inc., 123 Replicant Street, Los Angeles, California 90210--4321}}




% use for special paper notices
%\IEEEspecialpapernotice{(Invited Paper)}




% make the title area
\maketitle

% As a general rule, do not put math, special symbols or citations
% in the abstract
\begin{abstract}
  Although much research has been performed that evaluates the performance profile of applications that are used within the scientific domain of nuclear energy, there is a significant need to consider the coupling of physical simulations and how this might affect performance and scaling on modern supercomputing resources. Using the simplest methods that we could find that gave accurate results and representative performance profiles, we have developed three new mini-apps, Hot, Wet and Bright, which work across a structured mesh and have been written such that they can be coupled together in order to perform studies on their performance as a multi-physics application.

  We briefly discuss the individual applications, and then how they were coupled, before we explain the performance issues that we encountered and develop upon future work that will be required in order to achieve optimal implementations of the multi-physics application as a total unit (AWKWARD).
\end{abstract}

% no keywords

% For peer review papers, you can put extra information on the cover
% page as needed:
% \ifCLASSOPTIONpeerreview
% \begin{center} \bfseries EDICS Category: 3-BBND \end{center}
% \fi
%
% For peerreview papers, this IEEEtran command inserts a page break and
% creates the second title. It will be ignored for other modes.
\IEEEpeerreviewmaketitle

\section{Introduction}

Modern supercomputing has reached a scale where the core counts at the node level are increasingly significantly each year, and this has major implications for writing portable and performant code. In particular, many of the largest supercomputers in the world include heterogeneous devices, such as accelerators, which greatly increase the core count, and require that significant data-parallelism is exposed within scientific application's algorithms in order to perform well.

Accelerator devices like the Intel Xeon Phi processors, have greater numbers of cores than existing Intel CPUs, for instance the Knights Landing has between 64 and 72, each with 4 hardware threads, whereas the Intel Xeon Broadwell can have up to 22 cores per socket, with less chance that the 2 available hyperthreads will help performance for HPC applications. The IBM POWER8 CPU also exposes a large core count, with up to 12 cores per socket and 8 hardware threads for a total of 192 (DOUBLE CHECK THIS) threads in total.

Extending the discussion to GPGPUs, NVIDIA's Tesla K40 devices have 15 streaming multi-processors (SMX), each of which support 192 processing elements, for a total of 2880 CUDA cores. While those cores aren't not cores in the traditional sense, given that instructions are issued to warps which operate in lock-step, it is clear that a great deal of parallelism needs to be exposed in order to target them. More recently, NVIDIA released the Tesla P100, which has increased the number of SMXs to 56 and decreased the number of processing elements in each to 64 (two warps), for a total of 3584 CUDA cores in total.

Of course, this continual increase in the number of cores available on a node has major implications for the scalability of codes, and this is amplified by the increasing number of nodes that are hosting those devices. 

Our previous work has shown that there are issues with scaling many scientific applications past a certain point, regardless of their node-level performance, as communication becomes the limiting factor. It would appear that we are reaching a stage where the scalability of applications demands that the highly parallel nodes are programmed using shared memory programming, in order to reduce the number of independent ranks that need communication. 

The benefit of GPUs is that they offer good performance and can handle significant portions of a problem and require no partitioning within the devices. More recently supercomputing resources have started to include multiple accelerator devices on a single node, and this offers the opportunity for programmers to exploit massive node-level parallelism to improve the scalability of applications. Of course this requires that there is some suitably performant method for communicating within a node. The upcoming NVLINK technology that will be distributed by NVIDIA is intended to provide fast communications between the GPUs on a node and quite improved performance compared to PCIe when communicating with the host CPU. Our expectation is that if this technology can be used to its full extent, it should offer great potential improvements to the scalability of HPC applications. 

It is considered an onerous and complicated task to program GPUs and until recently there has been little traction within the community to port existing large-scale applications, even though some of the largest supercomputers in the world, for instance Titan, include both CPUs and GPUs. We have shown in our previous research that GPUs can be programmed effectively, in most cases, using existing programming models. We have shown that productivity can be an important deciding factor in the choice of programming model, and further showed that directive-based models can offer a highly productive interface, that can be portable given good implementations, whilst not significantly sacrificing performance.

Many of our findings and discussions have been limited to individual applications which have, in general, performed well on modern hardware, and with modern parallel programming models. However, they do not represent that true form of large-scale scientific applications, in that they are isolated instances of algorithms that have do not represent the multi-application hierarchies that scientists are truly interested in seeing results for.

Our intention is to use our new suite of applications, individually to expose performance optimisations for their respective algorithms, and as part of a coupled multi-physics system, in order to demonstrate the additional problems that arise when you aggregate applications in a realistic manner.

\section{Contributions}

As part of this research and throughout this paper we present a number of contributions:

\begin{itemize}
    \item We have developed three new mini-apps that serve as proxy applications for important scientific algorithms. A heat conduction application, a fluid dynamics application, and a Monte Carlo Particle Transport application.

\item We demonstrate our approach to coupling those applications, both physically and in terms of the computer science aspects including mesh decomposition strategies.

\item We present results for those algorithms in isolation on modern supercomputing devices: a NVIDIA Tesla P100, an Intel Xeon Phi Knights Landing processor, an IBM POWER8 CPU, and a Intel Xeon Broadwell.

  \item We present the results of running those applications at scale, on those same devices.

    \item We present early results for running the full multi-physics simulation at scale.
\end{itemize}

\section{Background}

In this section we will present some basic details about each application, in order to make clear what each application is intended to simulate. Our intention was to choose the simplest methods possible, whilst achieving a reasonable level of accuracy, and in particuler focused upon capturing the performance profile of their respective classes of algorithms.

In order to give a clear discussion, we will explain the initial development of each application prior to discussing our efforts to couple them together. This means that we will explain how each application was optimally developed, but with the caveat that this would likely have to change once coupling was required. We did not particularly consider the coupling up-front, instead opting to choose the best strategy for each application in isolation, maximising the potential outcomes of the coupling exercise.

\subsection{Hot - Heat Conduction via a CG Solver}

We have developed a simple conjugate gradient (CG) solver that implicitly solves the heat conduction equation in order to solve the problem within a reasonable time-frame with acceptable fidelity. Our implementation uses a standard preconditioner, but otherwise adopts the simplest CG approach. 

The conjugate gradient method is iterative, and reduces the error of a guessed solution by walking along the eigenvectors of the solution vector. We will not develop or explain the mathematics underpinning the CG solver, and direct interested readers to explore the wealth of existing literature \cite{} (PUT A LOAD OF CG REFERENCES IN). The only physical feature of the application is manifested in the density calculations, where the densities are stored at the cell centers and interpolated to the edges.

The best decomposition for this algorithm is essentially regular, minimising the surface area to volume of each rank to reduce communication overheads. In terms of boundary conditions we selected reflective, as this was simple, and improves the verifiability of the results. It is also useful when coupling to the other systems.

\subsection{Wet - Fluid Dynamic via Direct Lagrangian-Eulerian Flux Calculations}

Our fluid dynamics simulation maintains the simplest possible methods, whilst keeping second order in space for all dependent variables. We chose to stay first order in time because algorithms that are second-order in time are somewhat more complicated, and the benefit is primarily reduce wallclock time. Our initial instinct is that the relative runtimes of each of the applications was not particularly important at this stage, but may revisit this issue in a later publication.

The direct Lagrangian-Eulerian algorithm takes Euler's equations, and explicitly solves them across a staggered structured grid. The application uses simple smoothing to simulate artificial viscosity, with shock heating accounted for as part of the mechanical work update. For the mass flux calculations, a Van-Leer flux limiter is used to maintain a monotonic profile at shock boundaries, and second-order interpolations are performed for energy and momentum to ensure monotonic behaviour for all dependent variables.

Directional splitting is used in order to make the application two-dimensional, and we chose to maintain symmetry by alternating the directional calculations with each timestep. Our explicit timestep controls limit the timestep based on the artificial viscosity coefficients and stop any flux extending beyond a single cell.

As with the heat condution application, the boundary conditions we chose for this application are reflective. This is particularly useful for fluid dynamics as tracking of conservation is important in verifying the results.

\subsection{Bright - Monte Carlo Particle Transport}

Our transport application uses a Monte Carlo particle tracking method, making it quite different to the other two algorithms. Our initial effort split the problem into batches of particles, each of which had an individual history that was followed in a time-dependent manner. 

A major aspect of the application is the data structure used to describe particles. Given that we did not have prior experience with this particular algorithmic class, we were not exactly sure which of the potential approaches would end up being the most performant, so we explored the space and performed some testing which we describe in detail later. Ultimately, we described a single particle using a data structure that contains the particle's position in space, direction, energy, and initially tracked the particles location by cell on the mesh.

Our simple tracking procedure would maintained several running timesteps to different event types: (1) a Collision event, which is where the particle would either change direction or have some of its weight absorbed, (2) a Facet event, where the particle reached the boundary of a cell or zone, and (3) a Census event, where the particle reached the end of the current timestep. Of course, the census events for the application in isolation were simply set to some small constant size, with the intention for this timestep to be later controlled by the requirement of the CFL condition exposed by the fluid dynamics simulation.

In order to determine if a Collision event has occured, we perform a lookup to cross sectional data taken from the ENDF database. In our particular approximation, we only consider absorption and scattering, ignoring any contributions due to fission. In order to reduce the overall variance of the simulation, we extend the lifespan of particles by giving them a weight, and reducing this weight each time an absorption event occurs, rather than declaring them as dead particles. Once the weight has reduced past a certain point, or the particle has reached a low enough energy, we will only then consider the particle to be dead.

When a Collision event occurs and the result is a scattering of the particle a random number is generated to determine the angle of scattering, and this is also used to determine the level of energy dampening that occurs using well known relations \cite{}.

To perform facet intersection checking, we are able to leverage the simple geometry of the structured grid to calculate the new direction using a simple method of our own devising. We first check which of the edges would be hit first if the particle were travelling at the same speed but only along a single axis, and use this to solve the system of equations that arises to determine the intersection point. We have considered that this might not be optimal, and doesn't generalise beyond the basic geometry presented by our structured grid case.

In terms of boundary conditions we did select reflective at this stage, as this made it easier to track some conservation in the system, and matches the approach taken with the hydrodynamics solver, although we did recognise that it would likely be more representative if the boundary conditions were transmissive.

Of course, all of this requires a sufficient random number generator with a large period. In order to fulfil this requirement we leveraged an existing lightweight Mersenne-Twister algorithm, to generate random numbers for the whole system.

\section{Parallelisation}

Each of the applications was parallelised and tuned in isolation, firstly to target the CPU, and then to target other heterogeneous devices, such as NVIDIA GPUs. In all cases we used MPI + OpenMP 4.5, and MPI + CUDA. This introduced additional effort in the process, but allowed us to make some commentary about the ability for each of the programming models to support the requirements of the applications.

\subsection{Hot}

The algorithm does not expose any load imbalance, and the MPI communication was simply nearest neighbour halo exchanges, performed after each timestep. As each of the core kernels is essentially a simple linear algebra method, such as a dot product or sparse matrix-vector multiply, it was straightforward to parallelise with OpenMP and CUDA, given the extent of the data-parallelism exposed by the algorithms.

At each iteration an alpha and beta value is calculated, which needs to be distributed amongst all of the ranks. The implication is that each iteration requires two calls to MPI\_Allreduce, an obvious performance bottleneck at scale. 

\subsection{Wet}

As with the heat conduction application, the algorithm does not have any inherent load imbalance, and only requires nearest neighbour communication. The explicit nature of the algorithm means that multiple kernels are invoked in order, each of which was simplistic enough that parallelisation was trivial with OpenMP.

\subsection{Bright}

Unlike the other two applications, the Monte Carlo method exposes some load imbalance. There are a number of approaches to improving this problem, and we explored some of them to select the best for our particular set of requirements. 

In terms of parallelising the algorithm there were several factors to consider, for instance it is important to maintain good data locality in particular for vectorisation on CPUs and coalescense on GPUs. Our initial strategy was to perform a coarse-grained parallelism, splitting the particles into batches in order for them to be processed in parallel. Of course this has implications in terms of the memory layout as the particles are tracked through space.

\section{Performance Analysis}

Our performance analysis observes the performance that could be achieved across a range of HPC devices.

\subsection{Approach}

\subsection{Initial Performance}

%\begin{figure}[t!]
%\centering
%\includegraphics[width=0.5\textwidth]{initial-results.png}
%\caption{Compares the performance of Clang OpenMP~4.5 with CUDA for the suite of kernels (higher is better).}
%\label{fig-initial-results}
%\end{figure}

%\begin{table}[!t]
%\renewcommand{\arraystretch}{1.3}
%\caption{Comparing the Clang OpenMP~4.5 and CUDA registers.}
%\label{tab-registers}
%\centering
%\begin{tabular}{c||c||c||c||c}
%\hline
%\bfseries Application & \bfseries Clang & \bfseries CUDA & \bfseries Difference & \bfseries Occupancy\\
%\hline\hline
%Vec Add & 20 & 10 & 2 & 0.95 \\
%Vec Add Sqrt & 28 & 18 & 1.6 & 0.93 \\
%\hline
%\end{tabular}
%\end{table}

\section{Motivation}

This paper is a window into a set of work that is attempting to extend the success of mini-apps for evaluating modern architectures and programming environments. Our hope is that we can begin to capture the features of production scientific codes that have been yet unexplored. In particular the coupling of diverse algorithms, and how this affects their performance and scaling.

In order to limit our focus to the most important algorithms that might be encountered within scientific production codes, we will adopt the dwarves of parallel programming and plan to develop a new application that captures the computational and communicational profile of  

\section{Coupling}

As this application is developed by computer scientists and primarily focusses upon the evaluation of performance profiles on modern architecture, the exact physical solution to our test problems is not necessarily important. Of course, we are conscious of the fact that couplings of certain algorithms and for modelling particular scientific concepts could lead to different performance profiles.

We could see two different approaches to coupling the applications, the first of which involves directly coupling their constituent equations and solving the whole system at once. This approach has two important drawbacks: (1) the complexity of this approach is significant, especially considering that we plan to extend this project into a suite of seven algorithms, and (2) boiling the algorithms under a single solve wouldn't capture the potential difficulties of executing diverse routines together.

Given this, our choice of applications allows for a simple coupling stage that requires very little scientific knowledge to achieve a reasonable, albeit inaccurate, outcome. The inputs of the hydro package are density and energy, and momentum in each dimension, while the inputs for the diffusion package are density and temperature. As such we performed a simple conversion between internal energy and temperature between the packages, and ran them sequentially.

\section{Asynchronicity Among Packages}

Modern hardware presents a number of challenges and some important opportunities. The balance of different packages within a full application might allow new avenues for overlapping long communications or IO times, and even the potential for overlapping the compute of packages within the full application. We are interested to explore a number of these issues and expect that this could be a future direction for work but have not investigated that point in this research.

\section{Application Scaling}

As we are presenting two applications that have been developed from scratch, we provide some simple scaling studies to demonstrate that each application performs to an acceptable level. This data gives us some baseline for comparison as we consider the applications within a multi-algorithmic execution setting.



\section{Future Work}

Our immediate considerations for future work are to extend the applications to include more complicated physical processes. Of course, it is important that we maintain the lightweight and flexible nature of the applications such that rapid testing on modern architecture can be performed, however we do recognise that important physical requirements like handle three-dimensions, multiple materials, and complex unstructured geometries, would also benefit from being analysed in the manner presented in the research.

\section{Related Work}

(SPEAK TO W ABOUT THIS SECTION, WHAT IS ACCEPTABLE TO BE TIED TO THIS PAPER)

\section{Conclusion}

\bibliographystyle{IEEEtran}
\bibliography{IEEEabrv,multi}

\end{document}


